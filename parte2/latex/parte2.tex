\documentclass{article}
\usepackage{svg}
\usepackage{hyperref}
\usepackage{amsmath}
\usepackage{amsfonts} 
\usepackage{graphicx}
\usepackage[a4paper,  total={6.75in, 10in}]{geometry}
\usepackage[utf8]{inputenc}

\title{Animazione di una pandemia con un automa cellulare}
\author{Giulio Pastorello, Federico Gnudi}
\date{Novembre 2023}

\begin{document}

\maketitle

\section{Il Modello}

\hspace{\parindent}L'ispirazione principale per la seconda parte del progetto è 
stata il \textit{Game of Life} di Conway. Questo è il più famoso esempio di 
\textit{automa cellulare}: un insieme di celle posizionate in una griglia di forma 
predefinita, in cui lo stato di ogni cella varia all'avanzare del tempo e 
soddisfando alcune condizioni che dipendono dallo stato delle celle limitrofe. \\
Per simulare una pandemia  in una popolazione si è usato un modello simile a 
quello della prima parte. Le celle possono assumere quattro stati: 
\begin{itemize}
    \item Healthy (sani)
    \item Infected (malati)
    \item Healed (guariti)
    \item Dead (morti)
\end{itemize}
Gli individui della popolazione possono evolvere solo in una precisa direzione: \\
$manca schemnino$
Un soggetto sano può contrarre il virus e diventare infetto, mentre un infetto
può sopravvivere alla malattia e guarire oppure morire. I passaggi di stati 
vengono regolati da parametri probabilistici che cercano di dare una simulazione
realistica della pandemia. \\
\end{document}